\documentclass[12pt]{report}


\usepackage[utf8]{inputenc}
\usepackage{amsmath}
\usepackage{parskip}
\usepackage{graphicx}

\title{Assignment 3}
\author{Lasse Berglund : 48-159707}
\date{\today}

\begin{document}
\maketitle
\section*{Problem 1.}

Before applying the Simplex algorithm to the given LP we need to write it in Standard Form, which means we need to add slack variables to change the inequalities to strict equality signs.

The optimal solution is $\{ x_1 = 0, x_2 = 0, x_3 = 10000 \}$ which gives an optimal value of $10000$. It took $7$ iterations, the tableaux follow on the next page.


\begin{table}
\begin{center}
\begin{tabular}{ l l l l l }
  0 & $+ 100x_1$ & $+ 10x_2$ & + $x_3$ \\  \hline

  $x_4$ = & 1 & $-x_1$ \\ 
  $x_5$ = & 100 & $-20x_1$ & $- x_2$ \\ 
  $x_6$ = & 10000 & $-200x_1$ & $-20x_2$ & $-x_3$ \\ 

  \\
  % x1 -> 1
  100 & $+ 10x_2$ & + $x_3$ & $-100x_4$\\  \hline

  $x_1$ = & 1 &  $-x_4$ \\ 
  $x_5$ = & 80  & $- x_2$ & $-20x_4$ \\ 
  $x_6$ = & 9800  & $-20x_2$ & $-x_3$ & $-200x_4$ \\ 


  \\
  % x2 -> 20
  900 & $+ 100x_4$ & + $x_3$ & $-10x_5$\\  \hline

  $x_1$ = & 1 &  $-x_4$ \\ 
  $x_2$ = & 80  & $- x_5$ & $+20x_4$ \\ 
  $x_6$ = & 8200  & $-x_3$ & $-200x_4$ & $+20x_5$ \\ 

  \\
  % x4 -> 1
  1000 & $+x_3$ & $-100x_1$ & $-10x_5$ \\  \hline

  $x_4$ = & 1 &  $-x_1$ \\ 
  $x_2$ = & 100 & $- x_5$ & $-20x_1$ \\ 
  $x_6$ = & 8000  & $-x_3$ & $+200x_1$ & $+20x_5$ \\ 
  
  \\
  % x3 -> 8000
  9000 & $+100x_1$ & $+10x_5$ & $-x_6$ \\  \hline

  $x_4$ = & 1 &  $-x_1$ \\ 
  $x_2$ = & 100 & $- x_5$ & $-20x_1$ \\ 
  $x_3$ = & 8000  & $+200x_1$ & $+20x_5$ & $-x_6$  \\ 

  \\
  % x1 -> 100
  9100 & $+10x_5$ & $-x_6$ & $-100x_4$  \\  \hline

  $x_1$ = & 1 &  $-x_4$ \\ 
  $x_2$ = & 80 & $- x_5$ & $+20x_4$ \\ 
  $x_3$ = & 8200  & $-200x_4$ & $+20x_5$ & $-x_6$  \\ 

  \\
  % x5 -> 80
  9900 & $+100x_4$ & $-10x_2$ & $-x_6$  \\  \hline

  $x_1$ = & 1 &  $-x_4$ \\ 
  $x_5$ = & 80 & $- x_2$ & $+20x_4$ \\ 
  $x_3$ = & 9800  & $+200x_4$ & $-20x_2$ & $-x_6$  \\ 

  \\
  % x1 -> 1
  10000 & $-100x_1$ & $-10x_2$ & $-x_6$   \\  \hline

  $x_4$ = & 1 &  $-x_1$ \\ 
  $x_5$ = & 100 & $- x_2$ & $-20x_1$ \\ 
  $x_3$ = & 10000  & $-200x_1$ & $-20x_2$ & $-x_6$  \\ 

\end{tabular}
\end{center}
\end{table}


\newpage

\section*{Problem 2.}

\includegraphics{figure.pdf}

When applying the Simplex method we move from the origin $(0,0,0)$, to $(1,0,0)$.
Then to $(1,80,0)$, then to $(0,100,0)$, then up to $(0,100,8000)$.
Then to $(1,80,8200)$, followed by $(1,0,9800)$, and finally $(0,0,10000)$. \\

In other words, we move from the far lower left corner, counter-clockwise around the base, then up and clockwise around the top.

\section*{Problem 3.}

Presumably the addition of another variable extends the feasibility region to a four dimensional object, with $16$ basic solutions. If we think of the similar situation with two variables $x_1,x_2$ it takes three iterations. With three variables it takes seven iterations. Following this pattern, we should expect to have to run Simplex for $15$ rounds. In general, if we follow the pattern of these problems we should expect to have to run $2^k-1$ iterations of the Simplex method, where $k$ is the number of variables.

\section*{Problem 4.}

\subsection*{1. Finding the dual}

We want to find the dual of the following LP problem.

\textbf{Maximize}\\
  $3x_1 + x_2$\\
\textbf{subject to:}\\
  $ x_1 + 2x_3 \le 4 $ \\
  $ x_1 + x_2 + x_3\le 6 $ \\
  $ x_1,x_2,x_3 \ge 0 $ \\

Following the definition of the dual, we have:

\textbf{Minimize}\\
  $\textbf{y}^T \textbf{b} $\\
\textbf{subject to:}\\
  $ \textbf{y}^T A \ge \textbf{c}^T $ \\
  $ \textbf{y} \ge 0 $ 

where $A =
\begin{bmatrix}
  1 & 0 & 2 \\
  1 & 1 & 1 
\end{bmatrix}
\quad 
$

$
b =
\begin{bmatrix}
  4 \\
  6 
\end{bmatrix}
\quad 
$

$
y =
\begin{bmatrix}
  y_1 & y_2  
\end{bmatrix}
$ 

Which gives us the following LP problem as the dual:

\textbf{Minimize}\\
  $ 4y_1 + 6y_2 $\\
\textbf{subject to:}\\
  $ y_1 + y_2  \ge 3 $ \\
  $ y_2        \ge 1 $ \\
  $ 2y_1 + y_2 \ge 0 $ 

\subsection*{2. Running with SCIP}

When solving the original problem with SCIP we get the solution:
$ x_1 = 4, x_2 = 2, obj = 12$
and the dual
$ y_1 = 3, y_2 = 0, obj = 12$

As expected both problems yield the same optimal value.



\section*{Problem 5.}
\subsection*{1. Transform LP to Matrix Form}
Minimize: \\
$
(\textbf{b} = )
\begin{bmatrix}
  5 & 7 & 12
\end{bmatrix}
$
$
\begin{bmatrix}
  x_1 \\
  x_2 \\
  x_3
\end{bmatrix}
$\\

Subject to:
$
\begin{bmatrix}
  0 & 1 & 2 \\
  1 & 1 & 1 
\end{bmatrix}
$
$
\begin{bmatrix}
  x_1 \\
  x_2 \\
  x_3
\end{bmatrix}
$
$
\ge
( \textbf{c} = )
\begin{bmatrix}
  3 \\
  2
\end{bmatrix}
$

\subsection*{2. Determine the Dual}
Using the same procedure as in Problem 2 above, backwards we have that the Dual is:

Maximize $ \textbf{c}^T \textbf{y}$\\
subject to: $ \textbf{A}^T \textbf{y} \le \textbf{b}$\\
Where $\textbf{y}=[y_1,y_2]^T$.

Or in other words.

Maximize $ 3y_1 + 2y_2$\\
subject to: \\
$ 
  y2        \le 5\\
  y1 + y2   \le 7\\
  2y1 + y2 \le 12 $\\


\subsection*{3. Solve the Dual graphically}

\includegraphics[scale=0.5]{dual_plot}\\
Using the plot above we can see that the optimal solution is $y_1 = 2, y_2 = 5$ which gives an optimal value of $19$.

\end{document}