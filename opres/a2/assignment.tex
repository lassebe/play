\documentclass[12pt]{report}

\usepackage[utf8]{inputenc}
\usepackage{parskip}
\usepackage{graphicx}

\title{Operations Research -- Assignment 2 }
\author{Lasse Berglund : 48-159707}
\date{\today}


\begin{document}
\maketitle
\section*{1 -- Greedy Knapsack Strategy}
\subsection*{(a) Calculate the ratio for each item, and sort items in its decreasing order of the ratios}
By sorting the items by the ratio of their profit and weight we get the following order: \\

$x_2,x_1.x_5,x_3,x_4$

\subsection*{(b) Output of the strategy}
By selecting the items in the ordering given in (a) we get the resulting vector: \\
$(1,1,1,0,1)$ \\
That is we pick all items except $x_4$. \\
Which gives us an objective value of 12.

\subsection*{(c) Showing non-optimality of strategy}
Given the following input: \\
$max  \qquad 8x_1 + 7x_2$ \\
$s.t.\  \ \qquad 8x_1 + x_2 \le 8$ \\
$x_1,x_2 \in \{0,1\}$ \\

Our greedy strategy would say that $x_1$ has a ratio of 1, whilst $x_2$ has a ratio of 7. As such it would lead us to pick x2 which is clearly not optimal.


\newpage

\section*{2 -- Subjects to timeslots}
The graph described in the image can be mathematically described as a bipartite graph: \\
$ \{ \{v_e,v_j,v_p,v_c,v_m\} \cup \{v_2,v_3,v_4,v_5 \},\\
    \{(v_e,v_3),(v_e,v_5),(v_j,v_2),(v_j,v_4),(v_p,v_3),(v_c,v_2),(v_c,v_5),(v_m,v_4)\}\}$ \\
\\
At first we introduce variables $x_1,\ldots,x_8 \in \{0,1\}$, one for each edge in the graph, to be understood as follows. $x=1$ if the course represented by the left node is placed in the time slot represented by the right node. 
\\
\\
After that we add constraints, first to model that we must take at least two from Math, Physics, and Chemistry (c1).
\\
\\
We also need to model the fact that we can take at most one from English and Japanese (c2,c3).
\\
\\
Then we add constraints to model the fact that only one course can be put in each time-slot (c4-c7).
\\ 
\\
Finally we need to model the objective function, that is we want to maximize the number of courses that we take. That is simply the sum of the used edges.
\\
\\
With all this we get the following Integer Programming problem: \\
\begin{verbatim}
maximize
  x1+x2+x3+x4+x5+x6+x7+x8
subject to
  c1: x5+x6+x7+x8 >= 2
  c2: x1+x2 <= 1
  c3: x3+x4 <= 1
  c4: x1+x5 <= 1
  c5: x2+x7 <= 1
  c6: x3+x6 <= 1
  c7: x4+x8 <= 1

binary
  x1 x2 x3 x4 x5 x6 x7 x8
\end{verbatim}

SCIP gives us the solution $(0,0,0,0,1,1,1,1)$, that is $x_1,\ldots,x_4 = 0$ and $x_5,\ldots,x_8 = 1$. Which in natural language means we put physics in period 3, chemistry in period 2 and 5, and math in period 4.

\section*{3 -- Maximum Matching and Minimum Vertex Cover}
For maximum matching we have the following formulation:

\begin{verbatim}
maximize
  x1+x2+x3+x4+x5+x6+x7+x8
subject to
  x3+x4+x5 <= 1
  x6+x7+x8 <= 1

  x1+x2+x3 <= 1
  x4+x6 <= 1
  x5+x8 <= 1
binary
  x1 x2 x3 x4 x5 x6 x7 x8
\end{verbatim}

And for minimum vertex cover we have:

\begin{verbatim}
minimize
  y1+y2+y3+y4+y5+y6+y7+y8

subject to
  y1+y5 >= 1
  y2+y5 >= 1
  y3+y5 >= 1
  y3+y6 >= 1
  y3+y8 >= 1
  y4+y6 >= 1
  y4+y7 >= 1
  y4+y8 >= 1

binary
  y1 y2 y3 y4 y5 y6 y7 y8
\end{verbatim}

When solving them with SCIP we get two solutions for each with an objective value of 3. It turns out they are identical. 


\section*{4 -- IP formulation of Knapsack problem}

I decided to model the Knapsack problem in slide 13.


\begin{verbatim}
maximize
 4x1 + 3x2 + 2x3 + x4 + 3x5 
subject to
  2x1 + x2 +4x3 + 5x4 + 3x5 <= 10
binary
  x1 x2 x3 x4 x5
\end{verbatim}

When solving with SCIP we get the expected solution (1,1,1,0,1) as in the first problem in the assignment.

\end{document}