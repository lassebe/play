\documentclass{article}
\begin{document}

\section*{2.1}

\subsection*{2.19}
\begin{tabular}{l|l|l|l|}
\cline{2-4}
  $A_{1}$ & -3 & -21 & -7 \\ \cline{2-4}
  $A_{2}$ & -5 & 0 & 0 \\ \cline{2-4} 
  $A_{3}$ & 0 & -26 & -12 \\ \cline{2-4} 
\end{tabular}\\ 
The minimax regret act is A2. \\

\subsection*{2.20}
\begin{tabular}{l|l|l|l|}
\cline{2-4}
  $A_{1}$ & 0 & 0 & -4 \\ \cline{2-4}
  $A_{2}$ & -8 & -12 & 0 \\ \cline{2-4} 
  $A_{3}$ & -1 & -15 & -1 \\ \cline{2-4} 
\end{tabular}\\
The minimax regret act is A1. \\

\section*{2.2}

\subsection*{Decision table}
\begin{tabular}{l|l|l|l|}
\cline{2-4}
  $A_{1}$ & 6 & 5 & 8 \\ \cline{2-4}
  $A_{2}$ & 5 & 5 & 8 \\ \cline{2-4} 
  $A_{3}$ & 10 & 10 & 0 \\ \cline{2-4} 
\end{tabular}\\ \\

\subsection*{Regret table}
\begin{tabular}{l|l|l|l|}
\cline{2-4}
  $A_{1}$ & -4 & -5 & 0 \\ \cline{2-4}
  $A_{2}$ & -5 & -5 & 0 \\ \cline{2-4} 
  $A_{3}$ & 0 & 0 & -8 \\ \cline{2-4} 
\end{tabular}\\ 
\\A1 dominates A2, but the minimax regret rule evalutes them as equivalent, because they both get their maximum regret from A3.


\section*{2.3}

\begin{tabular}{l|l|l|l|}
\cline{2-4}
  $A_{1}$ & 10 & 100 & 150 \\ \cline{2-4}
  $A_{2}$ & 10 & 12 & 150 \\ \cline{2-4} 
\end{tabular}\\ 
\\A1 dominates A2, but since the Optimism-Pessimism rule only looks at the best and worst case, it evalutes them as equivalent.

\section*{2.4}
\subsection*{Decision table}
\begin{tabular}{l|l|l|l||l|}
\cline{2-5}
  & & & & EMV \\ \cline{2-5}
  $A_{1}$ & 1 & 15 & 15 & 31/3 \\ \cline{2-5}
  $A_{2}$ & 5 & 12 & 15 & 32/3 \\ \cline{2-5}
\end{tabular} \\

\subsection*{Transformed table}
We transform by mapping $1 \to 4$, and all other numbers to themselves. \\
\begin{tabular}{l|l|l|l||l|}
\cline{2-5}
  & & & & EMV \\ \cline{2-5}
  $A_{1}$ & 4 & 15 & 15 & 34/3 \\ \cline{2-5}
  $A_{2}$ & 5 & 12 & 15 & 32/3 \\ \cline{2-5}
\end{tabular}

\section*{2.5}
In the decision table, we calculate the Regret R, which is $u - Max$ for each $u$ in the table, where $Max$ is the maximum value in each column. \\ 
We transform by $a*u + b, a \geq 0$. \\
Then the Maximum for each column will be $a*Max + b$. \\
The new regret $R'$ will be $(a*u + b) - (a*Max + b) = a*(u-Max) = a*R$. QED.

\end{document}