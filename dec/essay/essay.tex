\documentclass{article}

\begin{document}
\section{The problem}
The three papers I will discuss in this essay all deal with the Backward Induction Paradox that arises in classical Game Theory. It shows up in a number of games, such as the Centipede Game introduced by Rosenthal (1982), the finitely repeated Prisoners Dilemma, and the Chain-Store game.
\\
\\
While the structure of the games causes the reasoning to be slightly different in each example, the main idea behind the paradox is an alternating step-wise reasoning from the penultimate state to the initial state, which is where the Backwards part of Backwards Induction comes from. It works under the assumption that both players are rational, and their rationality is common knowledge (that is each player knows that the other knows that she knows that ad infinitum). The paradoxical result of that reasoning is that the actions classical Game Theory advocate are both unintuitive and seem far removed from observed behaviour.
\\
\\
I will try to briefly summarise three papers that try to tackle this topic, spend some time comparing them, and finally give my own account of how I view the paradox and I how I believe it should be resolved. 





















\section{The papers}
\subsection{Pettit \& Sugden}
Pettit and Sugden set out to prove that the arguments that underly the Backward Induction-solution are unsound. They formulate the paradox as a repeated Prisoner's Dilemma, where the paradox is dependent on the rationality of the players, and common belief of rationality. That is, both players believe that the other is rational, and that the other believes that she believes etc. Given such a belief structure, the players are convinced that the other will always defect, because that will maximise expected pay-off.
\\
\\
The main point of contention, as far as I am able to tell, is that the Backward Induction argument ignores "past information". They argue that in if in the Prisoner's Dilemma, a player should ever cooperate, than she cleary invalidates the common belief of rationality because she diverged from the Backward Induction strategy of always defecting. Either the player who cooperated is irrational, or she believes that she can coerce the other to behave irrationally, put in other words, she does not believe that her opponent is rational. With this potential breaking in mind, they argue that neither player can be sure that the common belief in rationality will endure and without that surety the players are not able to complete the reasoning that leads to the Backward Induction solution. They also note, that in their opinion, even if both players believe that the common belief in rationality will survive no matter what happens in the game, it might not be sufficient for the Backward Induction solution to obtain.
\\
\\
Having, in their own words, solved the paradox, Pettit and Sugden turn to what they call the intuitive conclusion; wherein the players try to establish at least a temporary cooperation. They mention Tit-for-tat as an example of a strategy that might be more successful, but still cannot be regarded as rational. From there they expand Tit-for-tat into a strategy which is based on tricking the other into beliving that you are playing the irrational Tit-for-tat strategy, for as long as possible. They then present a number of examples of belief structures that a player can hold, which would make a delayed Tit-for-tat strategy a rational choice. The main idea here is to get the other to cooperate for a little bit longer than you do, so you both reap the rewards of cooperation, and you get a small extra pay-off for defecting earlier. However, as they are careful to explain, this does not mean that it is uniquely rational to cooperate initially. Instead, it depends on what beliefs are held by the player. 
\\
\\
They then bring up some counter-arguments, including the idea that all belief might be common. They explain that this might cause the players to be able to infer each others reasoning. But they argue that their argument still applies, even under these stronger conditions. They proceed to similarly defend their reasoning for delayed Tit-for-strategy. 
\\
\\
After that they turn to the situation where rationality is not only common belief but common knowledge. Here they admit that their argument falls apart. Common knowledge cannot break down in the same way that belief can. But Pettit and Sugden feel that the paradox is inapplicable in that case. Because in a situation where common knowledge of rationality holds it is uninteresting to ask oneself about a situation where one player cooperates, because such a situation can never arise. Because of this Pettit and Sugden dismiss common knowledge of rationality as something that "ought to have no interest for game theory", since it in their view leaves no room for any form of strategic thinking.















\subsection{Bicchieri}

Bicchieri instead shapes her paper around what see calls common knowledge of rationality by framing it around the question of how much knowledge about the others beliefs the players need for the Backward Induction solution to obtain. According to Bicchieri more information doesn't necessarily help the players to make better predicitions. At the point where the knowledge of the theory of the game becomes common knowledge, the theory of the game becomes inconsistent.
\\
\\
She defines consistency with regards to a theory of a game, as being free from contradiction given any information set, or state of the game (including previous actions). It's enough for one player to have an inconsistent theory, for the others to also become inconsistent, because their theories in turn are dependent on the theory of that player.
\\
\\
From there she claims that if common knowledge does indeed make players theories inconsistent, it means that common knowledge (of beliefs) can explain how players can play strategies that diverge from the one suggested by Backward Induction, and still remain individually rational.
\\
\\
Biccieri then demonstrates how the Backward Induction solution can obtain both when the players have the same set of beliefs and when their beliefs differ. Under the assumption that the players don't know what the other belives.
\\
\\
But what about the case when both players know what the other believes? When the beliefs are common knowledge. In this case Bicchieri brings up a game similar to the Centipede Game, and argues that if Player 1 chooses to continue the game (going against the Backward Induction Solution) the second player cannot think that this is because Player 1 believes her to be irrational, nor that Player 1 believes that she thinks he is irrational, this leaves the only option that Player 1 is irrational. But because beliefs are common knowledge, she knows that Player 1 believes himself to be rational, and this belief (given some additional assumptions) implies that he knows that he is rational, and one can only know true things. Bicchieri means that therefore there is no way for Player 2 to maintain her beliefs, either Player 1 is irrational, or he is trying to trick her (doesn't believe she is rational), and her theory is rendered inconsistent. 
\\
\\
She then goes on to talk about the communication of beliefs, and thereby making them common knowledge. This doesn't really play into the larger idea of the paradox, so I will refrain from describing it in greater detail.
\\
\\
Bicchieri clarifies that she is not trying to show how this can be used to figure out alternative strategies, but rather to show that despite the Backward Induction, players might rationally play these alternative strategies.

















\subsection{Aumann}
Aumanns paper is the latest of the three in chronological order. He begins by mentioning the idea of common knowledge of rationality, and cites both Bicchieri and Pettit as examples of how this idea is difficult to formalise. He states early on that his goal with the article is to formalise and then prove that "in Perfect Information games, common knowledge of rationality implies backward induction".
\\
\\
He proceeds to give very notation-heavy proofs of two theorems
$$ Theorem\ A. \ CKR \subset I $$
$$ Theorem\ B. \ For \ every \ PI\ game,\ there\ is\ $$
$$a\ knowledge\ system\ with\ \emptyset \neq CKR \subset I $$
where the first states that common knowledge of rationality causes the backward induction solution to obtain. And the second claims that every Perfect Information game there exists some set of knowledge for each player such that rationality becomes common knowledge.
\\
\\
After proving these theorems he gives some clarifications. Each player must consider each state of the game as though it has been reached, and she is only rational if she consider her actions in that state in the light of that state having been reached, she cannot ignore states because her knowledge of the game claims that such a state will never be reached. Aumann also stresses the temporal aspect of his interpretation of knowledge, he considers knowledge as what the players know before any moves have been played.
\\
\\
When putting the paper into the broader context of the literature, Aumann argues that whether backward induction and common knowledge of rationality are possible depends on what model one uses. And mentions that he does not take issue with the different interpretations that others have laid forth.
\\
\\
After some discussion he comes back to the idea of past play. The idea that a player takes into account the series of moves that led to a specific state in which she must make a decision. Aumann does not argue that the player cannot take into account these past moves, but simply says that she cannot ignore any state. In the example of the Centipede game, he argues that other solutions than Backward Induction may indeed be possible even if both players are rational. But, he reminds us, if that rationality is common knowledge, then by Theorem A, backward induction must obtain.
\\
\\
To close out his paper Aumann stresses the fact that he does not argue that common knowledge of rationality is likely to be the case. But that in Perfect Information games, it is always possible that it might. And if it should be the case, then according to his model, the backward solution must obtain. 



















\section{Compare and contrast}

Both Petit \& Sugden and Aulmann are in agreement about the fact that Backward Induction holds if rationality is common knowledge. Petit \& Sugden however feel that this is an uninteresting situation. Seeing it as the fact that the players are locked into the Backward Induction solution causes the hypothetical intuitive response to become irrelevant. Aulmann does however argue that a player has to consider each state of the game as though it has been reached. Bicchieri does not weigh in on common knowledge of rationality in her paper.
\\
\\
When it comes to common knowledge of beliefs both Bicchieri and Petit \& Sugden agree that Backward Induction will not obtain. Their arguments both hinge on the fact that a player is unable to hold certain beliefs at certain states of a game. And those beliefs are necessary in order to complete the reasoning required for Backward Induction. Aulmann seems content to leave belief out of his model entirely.
\\
\\
Aulmann explicitly decides against making any statements with regards to alternative strategies. Bicchieri considers the strategic benefits of communicating beliefs as to induce common knowledge of belief in order to open up the field for strategies that diverge from Backward Induction. Petit & Sugden spend a comparatively large portion of their paper to argue for the rationality of other strategies, such as delayed Tit-for-Tat. All authors however seem to be in agreement that there are situations (or models) that allow for other strategies without either player being irrational.





















\section{My own views}
I will approach the paradox by looking at certain sets of knowledge that the authors have discussed. For each such knowledge set I will discuss what I believe it implies for the Backward Induction paradox, and what that in turn says about the Game Theoretic framework, both in terms of descriptive and normative aspects.

\subsection{Common Knowledge of Rationality}

I find myself most aligned with the views of Petit \& Sugden. While Aulmann does provide a convincing proof that Backward Induction obtains given common knowledge of rationality. But much like Petit \& Sugden, I do not find that a very interesting situation, even though I accept Aulmanns proof that all Perfect Information games have information sets such that rationality is common knowledge. Aulmann himself states that common knowledge of rationality is "rarely met in practice". Given that common knowledge of rationality is a such strong assumption, I don't think that the fact that it implies Backward Induction in any way invalidates Game Theory as a good descriptive or normative model.

\subsection{Common Knowledge of Beliefs}


\subsubsection{}

\end{document}