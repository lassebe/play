\documentclass{article}

\begin{document}
\section{The problem}
All the papers deal with the Backward Induction Paradox that arises in classical Game Theory. It shows up in a number of games, such as the Centipede Game introduced by Rosenthal (1982), the finitely repeated Prisoners Dilemma, and the Chain-Store game.
\\
\\
While the structure of the games causes the reasoning to be slightly different in each example, the main idea behind the paradox is an alternating step-wise reasoning from the penultimate state to the initial state, which is where the Backwards part of Backwards Induction comes from. It works under the assumption that both players are rational, and their rationality is common knowledge %OR BELIEF DEPENDING ON WHO YOU ASK?
(that is each player knows that the other knows that she knows that ad infinitum). The paradoxical result of that reasoning is that the actions classical Game Theory advocate are both unintuitive and seem far removed from observed behaviour.
\\
\\
I'll try to briefly summarise three papers that try to tackle this topic, spend some time comparing them, and then give my own account of how I view the paradox and I how I believe it should be resolved. 

\section{The papers}
\subsection{Pettit \& Sugden}
Pettit and Sugden set out to prove that the arguments that underly the Backward Induction-solution are unsound. They formulate the paradox as a repeated Prisoner's Dilemma, where the paradox is dependent on the rationality of the players, and common belief of rationality. That is, both players believe that the other is rational, and that the other believes that she believes etc. Given such a belief structure, the players are convinced that the other will always defect, because that will maximise expected pay-off.
\\
\\
The main point of contention, as far as I'm able to tell, is that the Backward Induction argument ignores "past information". They argue that in if in the Prisoner's Dilemma, a player should ever cooperate, than she cleary invalidates the common belief of rationality because she diverged from the Backward Induction strategy of always defecting. Either the player who cooperated is irrational, or she believes that she can coerce the other to behave irrationally, that is, she doesn't believe that her opponent is rational. With this potential breaking in mind, they argue that neither player can be sure that the common belief in rationality will endure and without that surety the players are not able to complete the reasoning that leads to the Backward Induction solution. They also note, that in their opinion, even if both players believe that the common belief in rationality will survive no matter what happens in the game, it might not be sufficient for the Backward Induction solution to obtain.
\\
\\
Having, in their own words, solved the paradox, Pettit and Sugden turn to what they call the intuitive conclusion; wherein the players try to establish at least a temporary cooperation. They mention Tit-for-tat as an example of a strategy that might be more successful, but still cannot be regarded as rational. From there they expand Tit-for-tat into a strategy which is basically based on tricking the other into beliving that you are playing the irrational Tit-for-tat strategy, for as long as possible. They then present a number of examples of belief structures that a player can hold, which would make a delayed Tit-for-tat strategy a rational choice. The main idea here is to get the other to cooperate for a little bit longer than you do, so you both reap the rewards of cooperation, and you get a small extra pay-off for defecting earlier. However, as they are careful to explain, this does not mean that it is uniquely rational to cooperate initially. Instead, it depends on what beliefs are held by the player. 
\\
\\
















\subsection{Biccieri}
\subsection{Aumann}
Aumanns paper is the latest in terms of chronological order. He begins by mentioning the idea of common knowledge of rationality, and cites both Biccieri and Pettit as examples of how this idea is difficult to formalise.

\section{Compare and contrast}

\section{My own views}
I find myself sympathetic towards the

\end{document}