\documentclass{article}

\begin{document}
\section{The problem}
All the papers deal with the Backward Induction Paradox that arises in classical Game Theory. It shows up in a number of games, such as the Centipede Game introduced by Rosenthal (1982), the finitely repeated Prisoners Dilemma, and the Chain-Store game.
\\
\\
While the structure of the games causes the reasoning to be slightly different in each example, the main idea behind the paradox is an alternating step-wise reasoning from the penultimate state to the initial state, which is where the Backwards part of Backwards Induction comes from. It works under the assumption that both players are rational, and their rationality is common knowledge %OR BELIEF DEPENDING ON WHO YOU ASK?
(that is each player knows that the other knows that she knows that ad infinitum). The paradoxical result of that reasoning is that the actions classical Game Theory advocate are both unintuitive and seem far removed from observed behaviour.
\\
\\
I'll try to briefly summarise three papers that try to tackle this topic, spend some time comparing them, and then give my own account of how I view the paradox and I how I believe it should be resolved. 

\section{The papers}
\subsection{Pettit \& Sugden}
Pettit and Sugden set out to prove that the arguments that underly the Backward Induction-solution are unsound. They formulate the paradox as being dependent on common belief of rationality. 
But, they argue, that line of reasoning doesn't take into account anything that has happened up until that point.



\subsection{Biccieri}
\subsection{Aumann}
Aumanns paper is the latest in terms of chronological order. He begins by mentioning the idea of common knowledge of rationality, and cites both Biccieri and Pettit as examples of how this idea is difficult to formalise.

\section{Compare and contrast}

\section{My own views}
I find myself sympathetic towards the

\end{document}