\documentclass{article}

\begin{document}
\section{The problem}
All the papers deal with the Backward Induction Paradox that arises in classical Game Theory. It shows up in a number of games, such as the Centipede Game introduced by Rosenthal (1982), the finitely repeated Prisoners Dilemma, and the Chain-Store game.
\\
\\
While the structure of the games causes the reasoning to be slightly different in each example, the main idea behind the paradox is a step-wise reasoning from the penultimate state to the initial state, which is where the Backwards part of Backwards Induction comes from. The paradoxical result of that reasoning is that the actions classical Game Theory advocate are unintuitive and seem far removed from observed behaviour.
\\
\\
I'll try to briefly summarise three papers that try to tackle this topic, spend some time comparing them, and then give my own account of how I view the paradox and I how I believe it should be resolved. 

\section{The papers}
\subsection{Aumann}
Aumanns paper is the latest in terms of chronological order. 
\subsection{Petit}
\subsection{Biccieri}

\section{Compare and contrast}

\section{My own views}
I find myself sympathetic towards the

\end{document}