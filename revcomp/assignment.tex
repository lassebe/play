\documentclass[12pt]{report}


\usepackage[utf8]{inputenc}
\usepackage[T1]{fontenc}
\usepackage{amsmath}
\usepackage{parskip}
\usepackage{graphicx}

\title{Assignment}
\author{Lasse Berglund : 48-159707}
\date{\today}

\begin{document}
\maketitle

\section*{Program Inversion and Inverse Interpreter}
\subsection*{1.}
  In order to define a trivial program inverter, given an inverse interpreter, we simply take the given program $p$, we fix it in the inverse interpreter, then we simply feed the output $y$ into the combination.

  In lambda calculus: $\lambda p. \lambda y. [InvInt] (p,y)$.

\subsection*{2.}
  In order to define a trivial inverse interpreter, given a program inverter, we take the given program $p$, we feed it into the program inverter and get the inverse program $p^{-1}$. Then we simply run $p^{-1}$ with a normal interpreter on the output $y$.


  In lambda calculus: $\lambda p. \lambda y. [Int] ( [ProgInv] \ p ) \ y)$.


\section*{Reversible Programming}
The Janus program that 
\begin{verbatim}
n x1 x2

procedure fib
  if n=0 then x1 += 1
              x2 += 1
         else n -= 1
              call fib
              x1 += x2
              x1 <=> x2
  fi x1=x2

procedure main_fwd
  n += 4
  call fib


procedure main_fwd
  x1 += 4
  x2 += 8
  uncall fib

\end{verbatim}


\subsection*{3.} 
$ \{x1=0$, $x2=0$, $n=0 \} $ \\
  \textbf{ n += 4 }\\
$ \{x1=0$, $x2=0$, $n=4 \} $ \\
  \textbf{else n -= 1} \\
$ \{x1=0$, $x2=0$, $n=3 \} $ \\
  \textbf{call fib} \\
 \textbf{else n -= 1} \\
$ \{x1=0$, $x2=0$, $n=2 \} $ \\
  \textbf{call fib} \\
  \textbf{else n -= 1} \\
$ \{x1=0$, $x2=0$, $n=1 \} $ \\
  \textbf{call fib} \\
  \textbf{else n -= 1} \\
$ \{x1=0$, $x2=0$, $n=0 \} $ \\
  \textbf{call fib} \\
  \textbf{if n=0 then x1 += 1} \\
  \textbf{x2 += 1} \\
  \textbf{fi x1=x2} \\
$ \{x1=1$, $x2=1$, $n=0 \} $ \\
  \textbf{x1 += x2} \\
  \textbf{x1 <=> x2} \\
  \textbf{fi x1=x2} \\
$ \{x1=1$, $x2=2$, $n=0 \} $ \\
  \textbf{x1 += x2} \\
  \textbf{x1 <=> x2} \\
  \textbf{fi x1=x2} \\
$ \{x1=2$, $x2=3$, $n=0 \} $ \\
  \textbf{x1 += x2} \\
  \textbf{x1 <=> x2} \\
  \textbf{fi x1=x2} \\
$ \{x1=3$, $x2=5$, $n=0 \} $ \\
  \textbf{x1 += x2} \\
  \textbf{x1 <=> x2} \\
  \textbf{fi x1=x2} \\
$ \{x1=5$, $x2=8$, $n=0 \} $ \\

\subsection*{4.}
For readability I changed the addition statements to subtraction and vice-versa.


$ \{x1=5$, $x2=8$, $n=0 \} $ \\
\textbf{fi x1=x2} \\
\textbf{x1 <=> x2} \\
\textbf{x1 -= x2} \\
\textbf{uncall fib} \\
$ \{x1=3$, $x2=5$, $n=0 \} $ \\
\textbf{fi x1=x2} \\
\textbf{x1 <=> x2} \\
\textbf{x1 -= x2} \\
\textbf{uncall fib} \\
$ \{x1=2$, $x2=3$, $n=0 \} $ \\
\textbf{fi x1=x2} \\
\textbf{x1 <=> x2} \\
\textbf{x1 -= x2} \\
\textbf{uncall fib} \\
$ \{x1=1$, $x2=2$, $n=0 \} $ \\
\textbf{fi x1=x2} \\
\textbf{x1 <=> x2} \\
\textbf{x1 -= x2} \\
\textbf{uncall fib} \\
$ \{x1=1$, $x2=1$, $n=0 \} $ \\
\textbf{fi x1=x2} \\
\textbf{x2 -= 1} \\
\textbf{if n=0 then x1 -= 1} \\
$ \{x1=0$, $x2=0$, $n=0 \} $ \\
\textbf{else n += 1} \\
$ \{x1=0$, $x2=0$, $n=1 \} $ \\
\textbf{else n += 1} \\
$ \{x1=0$, $x2=0$, $n=2 \} $ \\
\textbf{else n += 1} \\
$ \{x1=0$, $x2=0$, $n=3 \} $ \\
\textbf{else n += 1} \\
$ \{x1=0$, $x2=0$, $n=4 \} $ \\


\subsection*{5.}
In the forward direction I interpreted the statements as follows:
\begin{itemize}
\item \textbf{ x += n } Add $n$ to the variable x.
\item \textbf{ x1 <=> x2 } Swap the values of x1 and x2.
\item \textbf{ if cond } If the boolean condition is true, execute the then branch, otherwise execute the else branch.
\item \textbf{ call prod } Execute the procedure prod.
\end{itemize}

In the backward direction I interpreted the statements as follows:
\begin{itemize}
\item \textbf{ x += n } Subtract $n$ to the variable x.
\item \textbf{ x1 <=> x2 } Swap the values of x1 and x2. (Unchanged)
\item \textbf{ fi cond } If the boolean condition is true, execute the then branch backwards, otherwise execute the else branch backwards.
\item \textbf{ uncall prod } Execute the procedure prod backwards.
\end{itemize}

\section*{Reversible gates}
\subsection*{1.}
  \subsubsection*{a.}
    To simulate a COPY gate we can simply set the top control to 1, and use the same structure as we would with a Feynman gate.

    \includegraphics[scale=0.5]{copy_gate.png}
  \subsubsection*{b.}
    In order to simulate an XOR gate, we fix one of the controls to 1, the other to a, and set the controlled to b.

    \includegraphics[scale=0.5]{xor_gate.png}
  \subsubsection*{c.}
    The classical way of synthesising an OR-gate given the building blocks we have so far would be to make use of the equivalence:
    $$ a \lor b = \neg ( \neg a \land \neg b) $$
  \subsubsection*{d.}


\subsection*{2.}


\end{document}